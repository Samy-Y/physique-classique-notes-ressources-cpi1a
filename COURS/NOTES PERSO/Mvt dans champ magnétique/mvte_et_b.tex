\document[class=report, fontsize=11pt, paper=a4, twoside=false]{scrartcl}
\usepackage[utf8]{inputenc}

\begin{document}

\section{Mouvement d'une particule chargée dans un champ électrique
et/ou magnétique uniforme et
permanent}\label{mouvement-dune-particule-charguxe9e-dans-un-champ-uxe9lectrique-etou-magnuxe9tique-uniforme-et-permanent}

\emph{Samy Youssoufine - Déc.-Jan.~2025}

\emph{Notes personnelles}

Il s'agit d'une nouvelle application du principe fondamental de la
dynamique (PFD) à une particule chargée en mouvement dans un champ
magnétique.

Nous allons définir deux champs : - Le champ statique
\(\overrightarrow{B}\) qui est \textbf{indépendant du temps}. On a donc
\(\frac{\partial \overrightarrow{B}}{\partial t}=0\). - Le champ
uniforme, c'est-à-dire que \(\overrightarrow{B}\) est constant en tout
point de l'espace, donc \textbf{indépendant de la position}
\(\overrightarrow{r}\).

Il y a donc une différence d'indépendance (temporelle/spatiale).

Nous allons maintenant étudier le mouvement d'une particule chargée dans
le champ électrique.

\subsection{Champ électrique statique et
uniforme}\label{champ-uxe9lectrique-statique-et-uniforme}

Pour créer un champ statique \(\vec{E}\), on utilise deux conducteurs
parcourus par un courant électrique, par exemple deux plaques
métalliques parallèles, auxquelles on va appliquer une différence de
potentiel \(U\).

On dit qu'il y a entre les deux plaques un \textbf{gradient de
potentiel}.

Le potentiel électrique va dépendre de la position \(x\) entre les deux
plaques selon la relation :
\[V(x) = \frac{V_+ - V_-}{d}x + V_+ = -Ex+V_+\] où \(d\) est la distance
entre les deux plaques, \(V_+\) et \(V_-\) sont les potentiels
respectifs des plaques, et \(E = \frac{V_+ - V_-}{d}=\frac Ud\) est
l'intensité du champ électrique entre les plaques.

L'énergie potentielle électrique d'une particule chargée \(q\) placée
entre les deux plaques est donnée par :
\[E_p(x) = qV(x) = q\left(-Ex + V_+\right) = -qEx + qV_+\]

Le résultat ci-dessous peut-être utilisé comme une définition.

Cela va créer un champ électrique \(\vec{E}\) entre les deux plaques, et
si on place une particule chargée entre ces plaques, elle va être
soumise à une force électrique \(\vec{F}_E = q\vec{E}\).

\begin{itemize}
\tightlist
\item
  C'est exactement analogue à la force gravitationnelle
  \(\vec{P} = m\vec{g}\). Ces deux forces ont le même caractère
  \textbf{conservatif}.
\item
  Si on abandonne une particule chargée \(q\) entre les deux plaques,
  elle va évoluer vers un état où son potentiel électrique est minimal,
  comme pour un objet lâché dans un champ de pesanteur.
\end{itemize}

En revanche, il faut noter que la charge \(q\) peut être négative. Dans
ce cas, la force électrique \(\vec{F}_E\) sera dirigée dans le sens
opposé au champ électrique \(\vec{E}\). C'est une différence importante
avec la force de pesanteur, qui est toujours dirigée vers le bas.

\subsection{Champ magnétique statique et
uniforme}\label{champ-magnuxe9tique-statique-et-uniforme}

Pour créer un champ magnétique statique et uniforme, on utilise une
bobine parcourue par un courant électrique continu. Une caractéristique
importante des bobines est que le champ magnétique produit va dépendre
du nombre de spires par unité de longueur \(n\) et de l'intensité du
courant \(I\) qui la traverse. On parle de \textbf{bobines de
Helmholtz}.

\subsection{Force de Lorentz}\label{force-de-lorentz}

Lorsqu'une particule chargée \(q\) se déplace dans un champ magnétique
\(\vec{B}\) et un champ électrique \(\vec{E}\), elle est soumise à une
force appelée \textbf{force de Lorentz}. Cette force est donnée par la
relation vectorielle suivante :

\[\vec{F} = q\left(\vec{E} + \underbrace{\vec{v} \wedge \vec{B}}_{\text{antisymétrique}}\right)\]

où \(\vec{v}\) est la vitesse de la particule.

\textbf{Effet symétrique du champ \(\vec{E}\) :} La force électrique
\(\vec{F}_E\) est dirigée dans le sens du champ électrique \(\vec{E}\).
La trajectoire va évoluer pour être alignée avec le champ électrique.

Trajectoire

Trajectoire

F\_e

F\_e

v\_0

v\_0

Text is not SVG - cannot display

\textbf{Effet antisymétrique du champ \(\vec{B}\) :} La force magnétique
\(\vec{F}_B\) est perpendiculaire à la fois à la vitesse \(\vec{v}\) de
la particule et au champ magnétique \(\vec{B}\).

{Note intéressante}

\textbf{Il existe 4 interactions fondamentales dans la physique} 1.
\textbf{Interaction gravitationnelle :} responsable de la force de
gravitation entre les masses. 2. \textbf{Interaction électromagnétique
:} responsable des forces entre les charges électriques et des
phénomènes magnétiques. 3. \textbf{Interaction faible (électrofaible) :}
responsable des processus de désintégration radioactive de type
\(\beta\), soit l'émission d'un électron ou d'un positron par un noyau
instable (\(N \to P + ne^+\) ou \(P\to N+ne^-\)). Quand un électron est
émis d'un noyau, l'interaction forte ne permet pas cette émission. Leur
ordre de grandeur est très faible, d'où l'appellation ``faible''. 4.
\textbf{Interaction forte :} responsable de la cohésion des noyaux
atomiques. Normalement, la répulsion entre protons devrait les faire
exploser, mais l'interaction forte les maintient ensemble. Leur ordre de
grandeur est très élevé, d'où l'appellation ``forte''.

{Équation du mouvement}

En appliquant le principe fondamental de la dynamique (PFD) à une
particule chargée \(q\) de masse \(m\) se déplaçant dans un champ
électrique \(\vec{E}\) et un champ magnétique \(\vec{B}\), on obtient
l'équation du mouvement suivante :

\[\frac{d\vec{v}}{dt} = \frac qm \left(\vec{E} + \vec{v} \wedge \vec{B}\right)\]

Cette formule reste valide dans le cas de la mécanique classique, tant
que les vitesses impliquées sont bien inférieures à la vitesse de la
lumière \(c=3\cdot 10^8 \, \text{m/s}\).

Seule \(\frac qm\), appelée \textbf{charge spécifique} ou \textbf{charge
massique}, est expérimentalement mesurable.

\textbf{Pour l'instant, nous allons nous limiter uniquement à l'étude
d'un système qui subit un champ électrique \(\vec{E}\), sans champ
magnétique \(\vec{B}\).}

{Remarque}

Nous n'allons étudier que le mouvement plan, donc quand \(\vec{v_0}\) et
\(\vec{E}\) sont dans le même plan.

Nous allons redéfinir rapidement le système étudié. Il s'agit d'une
particule chargée \(q\) de masse \(m\), de vitesse initiale
\(\vec{v_0}\).

Comme le champ \(\vec{E}\) est indépendant du temps, on peut intégrer
vectoriellement en prenant comme origine spatiale \(O\) le point où la
particule est lâchée à l'instant \(t=0\). On va aussi définir le plan
\((x,y)\) comme le plan dans lequel la particule évolue, qui contient le
vecteur vitesse initial \(\vec{v_0}\) et le vecteur champ électrique
\(\vec{E}\).

On obtient ainsi la relation :

\[\vec{v}(t) = \vec{v_0} + \frac qm \vec{E} t\]

Cela implique donc que
\(\vec{OM}(t)=\vec{v_0}t + \frac qm \frac{\vec{E} t^2}{2}\).

En projectant cette relation dans l'espace \((x,y,z)\), on obtient les
équations horaires suivantes :

\[\begin{cases}
x(t) = v_0\cos\alpha t + \frac qm \frac{E_x t^2}{2} \\
y(t) = v_0\sin\alpha t + \underbrace{\frac qm \frac{E_y t^2}{2}}_{=0} \\
z(t) = 0 \\
\end{cases}
\]

(On a pris \(\vec{E} = E\vec{u_x}\), donc \(E_y = 0\))

{Remarque}

Attention, ces équations horaires ne sont valables que lorsqu'on prend
l'origine spatiale \(O\) au point où la particule est lâchée à l'instant
\(t=0\). Il faut ajouter \(x_0\) et \(y_0\) si l'origine spatiale est
différente.

Nous allons maintenant étudier le mouvement de cette particule,
globalement.

\textbf{Le cas où \(v_{0y} = 0\) est trivial.}

En résumé, la particule va suivre une trajectoire rectiligne, accélérée
dans la direction du champ électrique \(\vec{E}\). Elle va d'abord se
diriger vers la direction du vecteur vitesse initial \(\vec{v_0}\), puis
va être déviée par le champ électrique \(\vec{E}\), et enfin va se
diriger dans la direction de ce champ électrique.

\textbf{Étude du cas général où \(v_{0y} \neq 0\)}

La vitesse initiale \(\vec{v_0}\) fait un angle \(\alpha\) avec le champ
électrique \(\vec{E}\). Elle n'est donc pas parallèle à ce champ.

On peut écrire l'équation de la trajectoire \(x(y)\) en éliminant le
paramètre temps \(t\).

On trouve ainsi :

\[x(y)=\frac{qE}{2mv_0^2\cos\alpha}y^2 + y\cot(\alpha)\]

La trajectoire suivie ici est une parabole (déviation parabolique).

Si on fixe quelques paramètres initiaux, on peut observer l'effet de la
position initiale sur la trajectoire de la particule chargée. En effet,
celle-ci restera la même, mais décalée en fonction de la position
initiale. Elle sera donc translatée.

\subsection{Aspects énergétiques}\label{aspects-uxe9nerguxe9tiques}

Le potentiel électrique \(V\) dans un champ électrique uniforme
\(\vec{E}\) varie linéairement avec la position \(x\) selon la relation
:

\[V(x) = -Ex + \text{constante}\]

L'énergie potentielle d'une charge \(q\) dans un champ électrostatique
\(\vec{E}\) est donnée par la relation :

\[E_p(x) = -qV\\\implies \boxed{E_p(x)=-qEx + \text{constante}}\]

(comme vu précédemment)

La force électrique \(\vec{F}_E = q\vec{E}\) est conservative, donc le
travail de cette force entre deux points \(A\) et \(B\) est égal à la
variation de l'énergie potentielle électrique entre ces deux points :

\[W_{A\to B}(\vec{F}_E) = E_p(A) - E_p(B)\]

{Démonstration}

\[\vec{f}=q\vec{E}=qE\vec{u_x}\]

\[\delta W(\vec{f}) = \vec{f} \cdot d\vec{OM}=qE\cdot dx\]

\[\delta W(\vec{f}) = -dE_p\]

\[\dots \text{CQFD.}\]

On peut aussi appliquer le théorème de l'énergie cinétique (TEC) entre
deux points \(A\) et \(B\) :

\[\Delta E_c = W_{A\to B}(\vec{F}_E) \\\implies E_c(B) - E_c(A) = qE(x_B-x_A)\]

\textbf{Attention : Il faut faire attention au signe de la tension et de
la charge pour bien définir les positions etc. etc. etc.} Si la
particule étudiée a une charge négative, elle va se diriger dans le
``sens croissant du potentiel électrique'', soit la plaque positive.
Dans le cas contraire, elle va se diriger dans le ``sens décroissant du
potentiel électrique'', soit la plaque négative.

On peut développer l'expression du théorème de l'énergie cinétique (TEC)
pour obtenir :

\[\Delta E_c = qE(x(t_2)-x(t_1))\]

\[\implies \Delta E_c=qE\left(v_0\cos\alpha (t_2 - t_1) + \frac qm \frac{E(t_2^2 - t_1^2)}{2}\right)\]

\[\implies \boxed{\Delta E_c = \frac{q^2 E^2}{m} \frac{(t_2^2 - t_1^2)}{2} + qE v_0\cos\alpha (t_2 - t_1)}\]

On étudie ensuite le signe de \(\Delta E_c\) pour déterminer si la
particule accélère ou décélère.

\textbf{Applications : accélérateurs linéaires}

\subsection{\texorpdfstring{Champ magnétique
\(\vec{B}\)}{Champ magnétique \textbackslash vec\{B\}}}\label{champ-magnuxe9tique-vecb}

\end{document}